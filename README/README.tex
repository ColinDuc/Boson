\documentclass[11pt, oneside]{article}   	% use "amsart" instead of "article" for AMSLaTeX format
\usepackage{geometry}                		% See geometry.pdf to learn the layout options. There are lots.
\geometry{letterpaper}                   		% ... or a4paper or a5paper or ... 
%\geometry{landscape}                		% Activate for rotated page geometry
%\usepackage[parfill]{parskip}    		% Activate to begin paragraphs with an empty line rather than an indent
\usepackage{graphicx}				% Use pdf, png, jpg, or eps§ with pdflatex; use eps in DVI mode
								% TeX will automatically convert eps --> pdf in pdflatex		
\usepackage{amssymb}
\usepackage{amsmath,amssymb}
\usepackage{epsfig,graphicx,latexsym,mathdots}
\usepackage{graphicx,amsfonts,dsfont,scalerel}
%\usepackage{hyperref}
\usepackage{amsthm}
\usepackage{amsmath}
\usepackage{framed}
\usepackage[colorlinks]{hyperref}
\usepackage{picture}
\usepackage{wrapfig}
\usepackage{float} 
\usepackage{caption}
\usepackage{subcaption}
\usepackage{listings}
\usepackage[title]{appendix}
\usepackage{upquote}

%SetFonts
\title{README : Searching for the Boson}
\author{Ducommun Colin, Elelamy Rayan, Van de Velde Anne-Sophie}
\date{}

\begin{document}
\maketitle

\section{Prerequisites}

The code is written in Python3, and makes extensive use of Numpy, the fundamental package for scientific computing in Python. Another external module used is \lstinline" csv " - it intervenes in importing ghe flow of data to and from the Python environment, in order to read the data from a \lstinline" .csv " file and to write the predictions in a \lstinline" .csv " file respectively. Therefore the user must have both packages to run this code. For installation information, please refer to the Python documentation. 

\section{Executing the code}

Extract the \lstinline" .zip " and open the terminal at the location of the extracted files. Then enter: \lstinline" python.run.py ". The file \lstinline" run.py " contains an executable Python script that takes train data to compute a model. To see more details about the objective of this scrpit, one should read our report. The desired train and tests sets are to be put in a \lstinline" train.csv " and \lstinline" test.csv ", which are provided as placeholders. 

\section{Description of the code}

The script first loads the data and initializes the hyperparameters. The values provided initially coincide with the optimal values mentioned in the report. 

First missing values are treated by deleting the columns where they occur. Then the script splits the training and test data into $3$ categories according to PRI\_jet\_num (see justification in the report). Then, a polynomial extension up ti the value of "degree" fallows. Finally, pairwise products of some if the features are performed. After this treatment, data is normalized and put in an appropriate format for the learning process.

The initial weights $w$ are random. The training is done using a regularized logistic regression, with parameter lambda. The optimal $w$ is approached via an iterative method: the gradient descent (see justification in report). The method outputs the optimal weights $w$ and the loss that corresponds to it.

Then, the test data goes through the treatment described above: from replacing missing values, to standardization. The predictions are performed then converted to the appropriate format $ \{ -1, 1 \}$ according to the best threshold (found via Kaggle). This process is repeated for each category of PRI\_jet\_num values $(0,1,2,3)$. The predictions are then rearranged to match the order of the original sample and put out using the \lstinline" csv " module. They can be found in the same folder as the \lstinline" run.py " script.

\section{implementations.py} \label{impl}

This file contains the methods that were to be implemented : 
\begin{itemize}
\item[i)] \lstinline" least_squares_GD(y, tx, initial_w, max_iters, gamma) ": performs a linear regression using a gradient descent algorithm. 
\item[ii)] \lstinline" least_squares_SGD(y, tx, initial_w, batch_size, max_iters, gamma) ": performs a linear regression using a stochastic gradient descent algorithm 

\item[iii)] \lstinline" least_squares(y, tx) ": performs a least squares regression using normal equations.
\item[iv)] \lstinline" ridge_regression(y, tx, lambda_) ": performs a ridge regression using normal equations.
\item[v)] \lstinline" logistic_regression(y, tx, initial_w, max_iters, gamma) ": performs a logistic regression using gradient descent.
\item[vi)] \lstinline" reg_logistic_regression(y, tx, initial_w, max_iters, gamma, lambda_) ": performs la regularized logistic regression using a gradient descent.
\end{itemize}
The inputs, when present, are : 
\begin{itemize}
\item \lstinline" y " which is a vector of flags (here it is an array of $-1$'s and $1$'s); 
\item \lstinline" tx " which is the matrix of features (in an order corresponding to \lstinline" y ");
\item \lstinline" initial_w " which is the starting weight for the iterative methods;
\item \lstinline" max_iters " which sets the maximum number of iterations for the iterative methods;
\item \lstinline" gamma " which is the step size for the iterative methods;
\item \lstinline" lamda_ " which is the factor of penalization in the ridge regressions;
\end{itemize}

All of these methods return the predicted weights $w$ and the loss. They are to be used cautiously and basic knowledge in machine learning is assumed.

\section{utilities.py}

This file contains auxiliary methods that are used in the methods from \ref{impl}. One can find here the methods used to standardize, to calculate the gradient, to build polynomials from a matrix $x$. It contains also the methods used to manipulate the data i.e. the one used for cross-validation, to add new features by multiplying the old ones and to separate them into four samples.

One can also find a method \lstinline" definitive_res_logistic(x, threshold) " that returns the decision $-1$ or $1$ from our matrix $x$ if the regression ligne is under or above a certain threshold.

\section{proj1\_helpers.py}

This file contains auxiliary routines that would otherwise make the script less readable. For most, they are of no interest by themselves and perform only data treatment like reading/writing from/in a \lstinline" csv " file.


\end{document}  